 \documentclass[a4paper,10pt]{article}
\usepackage[utf8]{inputenc}
\usepackage{amsmath}
\usepackage{amsfonts}
\usepackage{amssymb}
\usepackage{algorithm}
\usepackage[noend]{algpseudocode}
\usepackage{program}
\usepackage{graphicx}
\usepackage[T1]{fontenc}
\usepackage{eso-pic}
\usepackage{gensymb}
\usepackage{listings}
\usepackage{float}
\usepackage{upquote}
\usepackage{color}
\definecolor{bluekeywords}{rgb}{0.13,0.13,1}
\definecolor{greencomments}{rgb}{0,0.5,0}
\definecolor{redstrings}{rgb}{0.9,0,0}


\newcommand{\BackgroundPic}{\put(-4,0){\parbox[b][\paperheight]{\paperwidth}{\centering\includegraphics[width=\paperwidth,height=\paperheight]{nat-farve.pdf}}}}

\lstdefinelanguage{FSharp}
    {morekeywords={let, new, match, with, rec, open, module, namespace, type, of, member, % 
        and, for, while, true, false, in, do, begin, end, fun, function, return, yield, try, %
        mutable, if, then, else, cloud, async, static, use, abstract, interface, inherit, finally, default, override},
    otherkeywords={ let!, return!, do!, yield!, use!, var, from, select, where, order, by, None, Some},
    keywordstyle=\color{bluekeywords},
    sensitive=true,
	    breaklines=true,
    xleftmargin=\parindent,
    tabsize=4,
    morecomment=[l][\color{greencomments}]{///},
    morecomment=[l][\color{greencomments}]{//},
    morecomment=[s][\color{greencomments}]{{(*}{*)}},
    morestring=[b]",
    stringstyle=\color{redstrings}
    }

\lstnewenvironment{fslisting}
    {\lstset{language=FSharp,
                basicstyle=\ttfamily,
                breaklines=true,
                columns=fullflexible
    }}
    {}

\algnewcommand\True{\texttt{true}\space}
\algnewcommand\False{\texttt{false}\space}
\algdef{SE}[SUBALG]{Indent}{EndIndent}{}{\algorithmicend\ }%
\algtext*{Indent}
\algtext*{EndIndent}

\begin{document} 
	\AddToShipoutPicture*{\BackgroundPic}
	
	\begin{titlepage}
		\thispagestyle{empty}
		\vspace*{5cm}
		\begin{center}
			\Huge \textbf{Programmering og Problemløsning} \\
			\LARGE \textbf{Aflevering 10g - The Animal Race} \\
		\end{center}
		\vspace*{3.5cm}
		\begin{flushleft}
			
		\begin{table}[h!]
			\begin{tabular}{lll}
				Adam Ingwersen,& \\
				Aske Fjellerup,& \\
				Peter Friborg\\
			\end{tabular}
		\end{table}
			
			
			\vspace{3mm}
			\vspace{3mm}
			Datalogisk  Institut\\
			Københavns Universitet\\
			\vspace{3mm}
			\today\\
			%\vspace*{0.5cm}
			
		\end{flushleft}
	\end{titlepage}

	\title{10g}
	\author{AAP}
	
	\newpage

\newpage
\section{Introduktion}
I denne opgave stiftes der bekendskab med nedarvning i klasser og overskrivning af metoder. 

\section{10g.0}
Der startes med at opføres et UML-diagram, der giver et overblik over følgende opgave. Diagrammet skal have en \texttt{base class}, der tager to argumenter som \texttt{constructor}: \texttt{Animal(weight, maxspeed}. Der skal være to \texttt{derived} klasser: \texttt{Herbivore(weight, maxspeed)} \& \texttt{Carnivore(weight, maxspeed)}.\\De skal begge have to forskellige afarter af \texttt{Hunger()}, der beskriver hvor meget mad der skal spises iforhold til deres vægt. 
\begin{figure}[H]
\includegraphics[width=\textwidth]{UML.pdf}
    \centering
    \caption{UML-diagram over Animal og dens børn: Herbivore og Carnivore}
    \label{fig:my_label}
\end{figure}

\subsection{Animal}
Der laves her en klasse udfra ovenstående \texttt{baseclass}: \texttt{Animal}. \\
Der startes med at defineres alle attributer i klassen animal: this.Food skal gøres overridable, da det skal kunne ændres hvor meget hvad hvert dyr skal spise efter om det er et kød eller planteædende dyr.
\begin{figure}[H]
\lstset{language=FSharp}  
\begin{lstlisting}
  member this.Weight = Weight
  member this.MaxSp  = MaxSp
  member this.CurSp  = speed
  abstract member Food: float
  default this.Food   = food
\end{lstlisting}
    \centering
    \caption{Attributer i Animal}
    \label{fig:my_label}
\end{figure}
Der skal også laves metoder tilsvarende til UML-diagrammet. 
Disse skal den skal kunne arbejde med attributterne. \\
Den ene metode, \texttt{Hunger()}, skal laves som en \texttt{abstract member}, da den skal overskrives i underklasserne.\\
Koden vil være som følger:
\begin{figure}[H]
\lstset{language=FSharp}  
\begin{lstlisting}
  abstract member Hunger: unit-> unit
  default this.Hunger() =
    food  <- (Weight/2.0)
  member this.Eat Food =
    speed <- ((Food/this.Food)*this.MaxSp)
  member this.Run =
    let temp = (( float(RNG.Next (1,101)) ) / 100.0 )*this.Food
    this.Eat temp
    printfn "The animal eats %A kg and run for %A hours" temp (int(10000.0/this.CurSp))
\end{lstlisting}
    \centering
    \caption{Metoder i Animal}
    \label{fig:my_label}
\end{figure}
Det ses at metoden \texttt{Hunger()} er en abstrakt metode, og den kan altså overskrives i \texttt{Carnivore} og \texttt{Herbivore}. 
\subsection{Carnivore \& Herbivore}
Disse to underklasser skal arve alle ikke private \texttt{attributes} og \texttt{methods}. Dette skrives i koden:
\begin{figure}[H]
\lstset{language=FSharp}  
\begin{lstlisting}
  inherit Animal(Weight,MaxSp)
\end{lstlisting}
    \centering
    \caption{Nedarvning fra Animal}
    \label{fig:my_label}
\end{figure}
Nedarvningen kan dog ikke ændre på de mutable værdier der findes i \texttt{baseclass}, så derfor må der laves nogle værdier i begge tilfælde af \texttt{derived class}.
\begin{figure}[H]
\lstset{language=FSharp}  
\begin{lstlisting}
  let mutable food = 0.0
  let temp = Weight
\end{lstlisting}
    \centering
    \caption{Tilføjning af værdier til underklasserne}
    \label{fig:my_label}
\end{figure}\newpage
Der skal også overskrives en \texttt{attribute} og en \texttt{method}, og dette gøres udfra følgende:
\begin{figure}[H]
\lstset{language=FSharp}  
\begin{lstlisting}
  override this.Food = food
  override this.Hunger() =
    food <- (temp*0.08)
\end{lstlisting}
    \centering
    \caption{Her sker vores overskrivninger.}
    \label{fig:my_label}
\end{figure}
Ovenstående eksempel af overskrivning er fra \texttt{Carnivore}, det ses at den skal spise 8\% af sin vægt. Hvor \texttt{Herbivore}, skal spise 40\%.

\subsection{Test af program}
Programmet testes ved at der laves 3 opjecter (cheetah, antelope og Wilderbeast) som kører metoden \texttt{Run}. Dette sker 3 gange hvor efter dyret med den bedste tid vinder. \texttt{Run} er en metode der ud fra et tilfældigt procent vil køre de 2 metoder der fra opløget og derefter beregn og printer hvor lang tid det vil tage dyret at rejse 10km.
\\
\\
\textbf{Første gennemløb}
\\
\texttt{Cheetah}: 89 procent, der spises 3.56 kg og løbes 98 timer\\
\texttt{Antelope}: 94 procent, der spises 18.8 kg og løbes 111 timer\\
\texttt{Wilderbeast}: 92 procent, der spises 73.6 kg og løbes 135 timer
\\
\\
\textbf{Andet gennemløb}
\\
\texttt{Cheetah}: 38 procent, der spises 1.52 kg og løbes 230 timer\\
\texttt{Antelope}: 77 procent, der spises 15.4 kg og løbes 136 timer\\
\texttt{Wilderbeast}: 5 procent, der spises 4.0 kg og løbes 2500 timer
\\
\\
\textbf{Tredje gennemløb}
\\
\texttt{Cheetah}: 66 procent, der spises 2.64 kg og løbes 132 timer\\
\texttt{Antelope}: 88 procent, der spises 17.6 kg og løbes 119 timer\\
\texttt{Wilderbeast}: 68 procent, der spises 54.4 kg og løbes 183 timer
\\
\\
Resultatet er at dyret med det bedste gennemsnit er Antelope.
\end{document}

