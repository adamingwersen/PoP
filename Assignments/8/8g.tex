\documentclass[a4paper,10pt]{article}
\usepackage[utf8]{inputenc}
\usepackage{amsmath}
\usepackage{amsfonts}
\usepackage{amssymb}
\usepackage{algorithm}
\usepackage[noend]{algpseudocode}
\usepackage{program}
\usepackage{graphicx}
\usepackage[T1]{fontenc}
\usepackage{eso-pic}
\usepackage{gensymb}
\usepackage{listings}
\usepackage{float}
\usepackage{upquote}
\usepackage{color}
\definecolor{bluekeywords}{rgb}{0.13,0.13,1}
\definecolor{greencomments}{rgb}{0,0.5,0}
\definecolor{redstrings}{rgb}{0.9,0,0}


\newcommand{\BackgroundPic}{\put(-4,0){\parbox[b][\paperheight]{\paperwidth}{\centering\includegraphics[width=\paperwidth,height=\paperheight]{nat-farve.pdf}}}}

\lstdefinelanguage{FSharp}
    {morekeywords={let, new, match, with, rec, open, module, namespace, type, of, member, % 
        and, for, while, true, false, in, do, begin, end, fun, function, return, yield, try, %
        mutable, if, then, else, cloud, async, static, use, abstract, interface, inherit, finally},
    otherkeywords={ let!, return!, do!, yield!, use!, var, from, select, where, order, by, None, Some},
    keywordstyle=\color{bluekeywords},
    sensitive=true,
	    breaklines=true,
    xleftmargin=\parindent,
    tabsize=4,
    morecomment=[l][\color{greencomments}]{///},
    morecomment=[l][\color{greencomments}]{//},
    morecomment=[s][\color{greencomments}]{{(*}{*)}},
    morestring=[b]",
    stringstyle=\color{redstrings}
    }

\lstnewenvironment{fslisting}
    {\lstset{language=FSharp,
                basicstyle=\ttfamily,
                breaklines=true,
                columns=fullflexible
    }}
    {}

\algnewcommand\True{\texttt{true}\space}
\algnewcommand\False{\texttt{false}\space}
\algdef{SE}[SUBALG]{Indent}{EndIndent}{}{\algorithmicend\ }%
\algtext*{Indent}
\algtext*{EndIndent}

\begin{document} 
	\AddToShipoutPicture*{\BackgroundPic}
	
	\begin{titlepage}
		\thispagestyle{empty}
		\vspace*{5cm}
		\begin{center}
			\Huge \textbf{Programmering og Problemløsning} \\
			\LARGE \textbf{Aflevering 8g - Mastermind} \\
		\end{center}
		\vspace*{3.5cm}
		\begin{flushleft}
			
		\begin{table}[h!]
			\begin{tabular}{lll}
				Adam Ingwersen,& \\
				Aske Fjellerup,& \\
				Peter Friborg\\
			\end{tabular}
		\end{table}
			
			
			\vspace{3mm}
			\vspace{3mm}
			Datalogisk  Institut\\
			Københavns Universitet\\
			\vspace{3mm}
			\today\\
			%\vspace*{0.5cm}
			
		\end{flushleft}
	\end{titlepage}

	\title{8i}
	\author{AAP}
	
	\newpage

\newpage

\section{Forord}

\section{Introduktion}

Spillet Mastermind fra 1970 er et kodeløsnings-spil, som beskæftiger to spillere; en kodestiller og kodeløser. Kodestillerens opgave er at vælge en kombination af 4 ud af 6 mulige farvede brikker, som kan opstilles i en vilkårlig rækkefølge. Hertil skal kødeløseren forsøge at gætte den præcise kombination ved brug af højest 10 gæt. Kodestilleren skal for hvert gæt indikere korrektheden af gættet - dette ved at anføre en hvid brik, hvis farven eksisterer i kodestillerens kode og en sort brik, hvis farven og positionen er korrekt. Denne spilkonstruktion vil vi i denne rapport forsøge at implementere digitalt. 

\section{Problemformulering}

Målet for denne rapport er at implementere spillet Mastermind ved anvendelse af funktionelle- samt imperative programmeringsteknikker i \textbf{F\#}. Spillet skal kunne spilles i fire versioner, hvor der varieres mellem problemstiller/problemløser, begge roller skal kunne spilles af menneske hhv. computer i samtlige kombinationer. Spillet skal kunne spilles i en terminal - altså uden grafisk brugergrænseflade. Med dette i mente, lægges der vægt på at konstruere en intuitiv og let brugerinteraktion. 

\section{Problemanalyse \& Design}

\section{Programbeskrivelse}

\section{Afprøvning}

\section{Diskussion og konklusion}

\section*{Bilag}


















\end{document}