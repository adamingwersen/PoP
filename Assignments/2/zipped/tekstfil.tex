\subsection*{2i.0}
Denne delopgave beskæftiger sig med metasyntaksen 'Extended Backus-Naur Form' (EBNF). Vi betragter her en EBNF der indeholder 4 tokens. 
\begin{enumerate}
\item{charLiteral kan antage enhver unicode værdi}
\item{stringLiteral kan antage enhver unicode værdi i citationstegn}
\item{operator kan udelukkende antage værdien '+'}
\item{expression kan enten antage værdi i form af stringLiteral eller stringLiteral efterfulgt af operator efterfulgt af expression}
\end{enumerate}


Expressions kan udtrykkes udelukkende af kombinationer af tokener - som i denne sammenhæng ikke har noget eksplicit indhold:
\newline
\begin{enumerate}
\item{Exp1 $\leftarrow$ stringLiteral}
\item{Exp2 $\leftarrow$ stringLiteral, operator, stringLiteral, operator, Exp1}
\item{Exp3 $\leftarrow$ stringliteral, operator, Exp2}
\begin{itemize}
\item[=]{stringliteral, operator, stringLiteral, operator, stringLiteral, operator, stringLiteral}
\end{itemize}
\end{enumerate}

Hver enkelt token er defineret ved en eller flere terminaler - som er antager værdier som f.eks. unicode karakterer. Terminaler kan sammensættes inden for reglementet defineret i hvert token. Givet betragtede EBNF har vi, at mulige kombinationer kan være et arbitrært antal adderede strenge. Af mulige kombinationer, er 3 anført i listen nedenfor:
\newline
\begin{enumerate}
\item{"Chika chika" $\rightarrow$ "Chika chika"}
\item{"What?"+"..."+"My name is..."+"Who?..."+[Expression 1] $\rightarrow$ "What?...My name is...Who?..Chika chika"}
\item{"Hi.. My name is.. "+[Expression 2] $\rightarrow$ "Hi.. My name is.. What?...My name is...Who?..Chika chika"}
\end{enumerate}

EBNF'en er ikke i stand til at arbejde med værdier, som ikke er defineret i et token. Eksempler på situationer, hvor sekvenser er ikke-gyldige er angivet nedenfor:
\newline
\begin{enumerate}
\item{stringLiteral, stringLiteral, operator, operator}
\item{"Hej"*"Hej"}
\item{amamdmawdjwjadjaw}
\end{enumerate}


\subsection*{2i.1}
\begin{center}
    \begin{tabular}{ | l | l | l | l |}
    \hline
    Decimal & Binær & Heximal & Oktal \\ \hline
    10 & 01010 & A & 12 \\ \hline
    21 & 10101 & 15 & 25 \\ \hline
    63 & 00111111 & 3f & 77 \\ \hline
    63 & 00111111 & 3f & 77 \\
    \hline
    \end{tabular}
\end{center}

\begin{itemize}
\item[Fra Decimal til Binær:]{Helt generelt, kan vi udlede den binære værdi for et decimal-tal ved at dividere decimal-tallet med 2. Herfra skriver vi fra venstre mod højre; når divisionen resulterer i et heltal, angives 0, ellers 1. $10/2 \rightarrow 0, 5/2 \rightarrow 1, 2/2 \rightarrow 0, 1/2 \rightarrow 1, 0/2 \rightarrow 0$}
\item[Fra Decimal til Hex:]{Heximal kan antage 16 unikke værdier, fra 0 til 15 - den 10. i rækken er A.}
\item[Fra Decimal til Octal:]{Octal kan antage 8 unikke værdier, fra 0-7. Vi kan dividere decimalen med 8 - herefter kan vi multiplicere med den komma-denoterede rest: $10/8 \rightarrow 1, rest: 0,25 \rightarrow 0,25 \cdot 8 \rightarrow 2$}
\item[Fra Binær til Decimal:]{En binær række er følger sumfølgen  $n^2$ fra højre mod venstre, således at den 4. binære værdi i rækken (fra højre mod venstre) er $2^4 = 16$ hvis 1, 0 hvis 0. Vi kan således udlede værdierne for hver enkelt position og tage summen til følgen: $10101: 1 \cdot 2^4 + 0 \cdot 2^3 + 1 \cdot 2^2 + 0 \cdot 2^1 + 1 \cdot 2^0 = 16 + 4 + 1 = 21$}
\item[Fra Binær til Hex:]{Her vil vi udnytte, at vi kender decimalværdien: $21/16 \rightarrow 1, rest: 0,3125 \rightarrow 0,3125 \cdot 16 = 5 \rightarrow 15$}
\item[Fra Binær til Octal:]{Her udnytter vi igen at decimalværdien kendes: $21/8 \rightarrow 2, rest: 0,625 \rightarrow 0,625 \cdot 8 = 5 \rightarrow 25$}
\item[Fra Hex til Decimal:]{$3 \cdot 16^1 + F (= 15) \cdot 16^0 = 63$}
\item[Fra Hex til Binær:]{Vi udnytter at vi kender decimalværdien: $63/2 \rightarrow 1, 31/2 \rightarrow 1, 15/2 \rightarrow 1, 7/2 \rightarrow 1, 3/2 \rightarrow 1, 1/2 \rightarrow 1 = 111111$ Vi har nu 6 1-taller - vi bør derfor angive to 0'er foran idet hexadecimal-til-binær antager 4 værdier. Disse resultater er ækvivalente}
\item Hex til Octal:]{Vi kender decimalværdien: $63/8 \rightarrow 7, rest: 0,875 \rightarrow 0,875 \cdot 8 = 7 \rightarrow 77$}
\end{itemize}
Den sidste række i tabllen: Vi ser, at tallene er ens - og et dobbeltcheck sikrer, at værdierne i række 3 korrekt udregnet. Jeg vil derfor opskrive udregningerne for række 4. 

\subsection*{2i.2}

To mulige løsninger på problemet er angivet nedenfor:

\lstinputlisting[language=FSharp]{2i2.fsx}
