\documentclass[a4paper,10pt]{article}
\usepackage[utf8]{inputenc}
\usepackage{amsmath}
\usepackage{amsfonts}
\usepackage{amssymb}
\usepackage{algorithm}
\usepackage[noend]{algpseudocode}
\usepackage{program}
\usepackage{graphicx}
\usepackage[T1]{fontenc}
\usepackage{eso-pic}
\usepackage{gensymb}
\usepackage{listings}
\usepackage{float}
\usepackage{upquote}
\usepackage{color}
\definecolor{bluekeywords}{rgb}{0.13,0.13,1}
\definecolor{greencomments}{rgb}{0,0.5,0}
\definecolor{redstrings}{rgb}{0.9,0,0}


\newcommand{\BackgroundPic}{\put(-4,0){\parbox[b][\paperheight]{\paperwidth}{\centering\includegraphics[width=\paperwidth,height=\paperheight]{nat-farve.pdf}}}}

\lstdefinelanguage{FSharp}
    {morekeywords={let, new, match, with, rec, open, module, namespace, type, of, member, % 
        and, for, while, true, false, in, do, begin, end, fun, function, return, yield, try, %
        mutable, if, then, else, cloud, async, static, use, abstract, interface, inherit, finally},
    otherkeywords={ let!, return!, do!, yield!, use!, var, from, select, where, order, by, None, Some},
    keywordstyle=\color{bluekeywords},
    sensitive=true,
	    breaklines=true,
    xleftmargin=\parindent,
    tabsize=4,
    morecomment=[l][\color{greencomments}]{///},
    morecomment=[l][\color{greencomments}]{//},
    morecomment=[s][\color{greencomments}]{{(*}{*)}},
    morestring=[b]",
    stringstyle=\color{redstrings}
    }

\lstnewenvironment{fslisting}
    {\lstset{language=FSharp,
                basicstyle=\ttfamily,
                breaklines=true,
                columns=fullflexible
    }}
    {}

\algnewcommand\True{\texttt{true}\space}
\algnewcommand\False{\texttt{false}\space}
\algdef{SE}[SUBALG]{Indent}{EndIndent}{}{\algorithmicend\ }%
\algtext*{Indent}
\algtext*{EndIndent}

\begin{document} 
	\AddToShipoutPicture*{\BackgroundPic}
	
	\begin{titlepage}
		\thispagestyle{empty}
		\vspace*{5cm}
		\begin{center}
			\Huge \textbf{Programmering og Problemløsning} \\
			\LARGE \textbf{Aflevering 7i} \\
		\end{center}
		\vspace*{3.5cm}
		\begin{flushleft}
			
		\begin{table}[h!]
			\begin{tabular}{lll}
				Adam Ingwersen,& \\
			\end{tabular}
		\end{table}
			
			
			\vspace{3mm}
			\vspace{3mm}
			Datalogisk  Institut\\
			Københavns Universitet\\
			\vspace{3mm}
			\today\\
			%\vspace*{0.5cm}
			
		\end{flushleft}
	\end{titlepage}

	\title{7i}
	\author{AAP}
	
	\newpage

\newpage

\section{Delopgave 7i.0}

\subsection{safeIndexIf}
Funktionen er bygget såldes, at denne tager et brugerinput. Ved korrekt indicering - returneres værdien af det x'te index i arrayet A. \\
\\
Givet den betragtede array af længde 6, vil enhver værdi, der ikke er mellem 0 og 6 karakteriseres som invalid indeksering - hertil returneres default-værdien for typen array. I det tilfælde, hvor input er af en anden type end \texttt{\textit{integer}}, bliver brugeren præsenteret med følgende fejlmeddelelse; \texttt{"System.FormatException: Input string was not in a correct format."}. 
\\
Fejlmeddelelsen kunne præsenteres pænere ved at kalde funktionen med en try/with struktur fremfor bare at bruge printfn - dog bliver der i opgaveformuleringen specifikt spurgt efter en funktion, der ikke anvender try/with. 

\subsection{safeIndexTry}
Denne funktion anvender try/with strukturen til at kaste en fejl ved brug af \texttt{\textit{failwith}}. Hvis input til x er af korrekt type, men ugyldig værdi, vil fejlen \texttt{"Index is of invalid size"} returneres. Hvis ikke input x er af korrekt type vil fsharp returnere samme fejlmeddelelse som \texttt{\textit{safeIndexIf}}. For \texttt{\textit{safeIndexTry}} har jeg valgt ikke at specificere typen for inputtet x - men grundet fsharp's type-genkendelse, vil denne ikke acceptere, at andre typer end \texttt{\textit{integer}} anvendes til at indicere et array. 

\subsection{safeIndexOption}
I \texttt{\textit{safeIndexOption}} anvendes igen try/with strukturen, men nu med \texttt{\textit{option}}-typerne \texttt{\textit{Some}} og \texttt{\textit{None}} samt \texttt{\textit{Option.get}}-argumentet. Dette resulterer i, at et \texttt{\textit{integer}} returneres ved gyldigt input til x - mens \texttt{\textit{null}} returneres hvis x er out-of-bounds. \texttt{\textit{Option.get}} anvendes udelukkende idet opgaveformuleringen dikterer, at ingen af funktionerne må returnere andet end heltal. 

Hvis ikke input x er af korrekt type vil fsharp returnere samme fejlmeddelelse som \texttt{\textit{safeIndexIf}}.

\section{Delopgave 7i.1}
Problemstillingen i denne opgave, løses ved at konstruere to hjælpefunktioner som det første. En funktion, der læser (\texttt{readFile}) og en funktion, der skriver (\texttt{writeFile}. Disse anvendes efterfølgende i \texttt{fileReplace}. Disse to funktioner anvender \texttt{streamReader}- hhv. \texttt{streamWriter}-typerne. Der kan nu læses og skrives - for at erstatte en streng med en anden streng i teksten, anvendes \texttt{Regex.Replace((text : string), (pattern : string), (replacement : string))} fra \texttt{\textit{System.Text.RegularExpressions}} name-spacet.

Konstruktionen anvendt i programmet bevirker, at uanset indholdet af \texttt{filename}, \texttt{needle} og \texttt{replace}, køres programmet. I det tilfælde, hvor indholdet af tekstfilen er tom, returneres bare endnu en tom streng - og denne skrives til den i forvejen tomme tekstfil. Såfremt \texttt{needle} ikke findes i tekstfilen, erstattes intet. \texttt{replace} vil kun blive skrevet til tekstfilen, hvis \texttt{needle} findes i tekstfilen.

Der er ikke anvendt eksplicit fejhåndtering - men en mulig udbygning kunne være, at prompte brugeren, hvis f.eks. ingen matches findes - eller hvis den indgående tekstfil er tom. 

\section{Delopgave 7i.2}
Til dette program, inviteres brugeren til at vælge et URL samt et søgekriterium. Programmet forløber som følger:

\begin{enumerate}
\item Interagér med brugeren via konsol : læs brugerinput
\item Hent url-stream på baggrund af brugers input
\begin{enumerate}
\item Sikr at URL'et er validt
\item Skriv \texttt{\textit{url}} til \texttt{\textit{string}}
\item Skriv \texttt{\textit{string}} til tekst-fil lokalt
\end{enumerate}
\item Verificér at filen er blevet lavet og eksisterer
\item Læs fil som \texttt{\textit{string}}
\item Interagér med brugeren via konsol : læs brugerinput
\item Udfør pattern match i indlæst fil på baggrund af brugers input
\end{enumerate}

\noindent Programmet kan potentielt set fejle efter 1. brugerinteraktion, såfremt brugeren indtaster et ugyldigt URL. I dette tilfælde kastes en fejl: \texttt{\textit{"Sitet kan ikke findes. Indtast venligst gyldig URL"}}. \\

\noindent Programmet har ligeledes mulighed for at fejle efter 2. brugerinteraktion, hvis brugeren indtaster en streng der returnerer 0 matches. Denne fejl håndteres ved at definere en exception og bruge try/with med printfn, som anvist i \texttt{\textit{fsharpNotes11-12.pdf}} under \texttt{\textit{Listing 11.5}}. \newline

\noindent Funktionen \texttt{\textit{countHtmlTag}} anvender namespacet: \texttt{\textit{System.Text.RegularExpressions}} for bekvemhed. Herunder funktionen \texttt{\textit{Regex.Matches}}, som både tager regular-expressions eller en simpel-tekststreng som søgekriterium. 














\end{document}